\documentclass[a4paper]{article}

\usepackage{longtable}

\usepackage[table]{xcolor}
\definecolor{myGray}{gray}{0.8}
\definecolor{emailGray}{gray}{0.1}

\usepackage{ifpdf}
\ifpdf
\usepackage[pdftex, colorlinks=true, urlcolor=emailGray]{hyperref}
\else
\usepackage[hypertex]{hyperref}
\fi


\hoffset=-50pt
\textwidth = 450pt
\voffset=-45pt
\textheight = 690pt

\linespread{1.1}

\begin{document}

\begin{center}
\Large{\bf{ANSIOLUETTELO}}
\end{center}

\begin{flushleft}
  \textbf{Ivan Martynov}\\
  Puh. 044 936 6589\\
  S\"ahk\"oposti: \href{ivan.a.martynov@gmail.com}{ivan.a.martynov@gmail.com}\\
  Huhtiniemenkatu 19 C 20, 53810 Lappeenranta, Finland\\
  Syntynyt: 12.04.1982, Petroskoi. Henkil\"otunnus: 120482--165E\\
  Kansalaisuus: Suomi, Ven\"aj\"a
\end{flushleft}
\vspace{-10pt}

\begin{longtable}{p{0.12\textwidth} p{0.88\textwidth}}
  \textbf{Tavoite:} &
  Min\"a etsin seuravaa paikkaa:
  \vspace{-8pt}
  \begin{itemize}
    \setlength\itemsep{-3pt}
    \item[-]ohjelmoija/kehitt\"aj\"a: voin tehd\"a moninaisia teht\"avi\"a
      (yleisohjelmointi, tutkimus, tietokannat jne.), kuitenkin kiinnostavimia
      alueet ovat peliohjelmointi (pelattavuus, pelimoottori ja
      yleisohjelmointi), tietokonegrafiikka ja kuvank\"asittely
    \item[-]tutkija: olen kiinnostunut tutkimusta matematiikan ja tietotekniikan
      alueella (kuvank\"asittely, ohjelmointi, data analyytikko, simulointi
      jne.)
    \item[-]olen my\"os kiinnostunut harjoittelijan paikasta
  \end{itemize}
  \vspace{-8pt}
Minulla on vankka matemaattinen tausta ja erilaisia kokemusta.

Matemaattinen ajatteleminen auttaa ratkaista erilaisia ep\"atriviaalia ongelmia.

Osaan koodata erilaisella kielell\"a: C/C++, Java, Python, Matlab. Tykk\"a\"an
ohjelmoiminen ja uuden kielen oppiminen ei ole ongelma minulle.

Minulla on tutkijan kokemus Lappeenrannan teknillinen yliopistosta: ohjelmointi,
kuvank\"asittely, ominaisuus l\"oyt\"aminen (esim., pilvet, vesi ja/tai varjot
geografisess\"a kuvissa).

Osaan Suomea melkein hyvin ja osaan Englantia sujuvasti.
\\
& \\
\multicolumn{2}{l}{\cellcolor{myGray}\textbf{Koulutus:}}\\
& \\

\textbf{7\,/\,2012\,--\,nykyisyys:} &
  \textmd{\textsl{Lappeenrannan Teknillinen Yliopisto, School of Engineering
  Science, Laskennallinen Tekniikka ja Fysiikka.}}

  \textit{P\"a\"a-- ja sivuaineet:} Kuvank\"asittely, ominaisuuiden tunnistus
  (esim., pilvet ja varjot maantieteellinen kuvoissa). Tutkimusty\"oss\"ani olen
  k\"aytt\"anyt erilaiset metodit huomata eri\"a maapohjan ominaisuuksia:
  \vspace{-10pt}
  \begin{itemize}
    \setlength\itemsep{-3pt}
    \item[-]Spektriin arvojen analyysi satelliitin kuvoissa: sopii pilvien,
      vesialueen, kasvillisuuiden ja muu maaston alueen tunnistukseen. Yleisesti
      k\"aytetty Landsatin satelliittikuvien kanssa;
    \item[-]K\"aytt\"a\"a maaston korkeuden arvoja ja solar kulmat tunnistaa
      varjoja.  Yleisesti k\"aytetty maantieteellinen korkeuden malleissa, esim.
      Shuttle Radar Topography Mission (SRTM) data.
  \end{itemize}
  \vspace{-10pt}
  Metodit auttavat arvioida millainen alueet (joka maastossa) voi n\"aytt\"a\"a
  h\"am\"ar\"a tunnettu ajankohdassa.

  Teknologian tohtorintutkinto (arvioiden suoritusaika: Joulukuu 2017).
\\
&\\
\textbf{9\,/\,2006\,--\,9\,/\,2008:} &
  \textmd{\textsl{Lappeenrannan Teknillinen Yliopisto, School of Engineering
  Science, Laskennallinen Tekniikka ja Fysiikka.}}

  \textit{P\"a\"a-- ja sivuaineet:}Tietotekniikka, academic writing in English,
  tilastollinen analyysi mallintamisessa, evolutiivinen laskeminen ym.

  Tarkastajat: prof. Heikki Haario ja dosentti Tuomo Kauranne.
  Diplomity\"o: \emph{Computing the persistent homology of range images with
  alpha shapes.}
  Teos on noin kolmiulotteisen pistepilvien k\"asitellyst\"a ja
  rakenteen analyysi: m\"a\"aritell\"a tietojen homologiaa riippuen pisteist\"a.
  Kehitetty metodit voidaan k\"aytt\"a\"a laskemaan Betti numerot ja ja n\"ain
  arvioimaan rakenneosien lukum\"a\"ar\"a pistepilvess\"a sek\"a yksi-- ja
  kaksiulotteisen kolojen lukum\"a\"ar\"a.\\

&\\
\textbf{9\,/\,2002\,--\,6\,/\,2008:} &
  \textmd{\textsl{Petroskoin valtionyliopisto, Matematiikan laitos, Topologian
    ja Geometrian koulu.}}

  \textit{P\"a\"a-- ja sivuaineet:} Matematiikan analyysi, kombinatoriikka,
  funktioanalyysi, topologia, differentiaaliyht\"al\"ot, teoreettinen mekaniikka
  ym.

  Tarkastaja: prof. Aleksandr Ivanov.
  Diplomity\"o: \emph{About free products homeomorphisms.}
  Teos on noin topologinen homeomorfismin teorian kehitt\"aminen erikoisessa
  topologisessa rakenteissa.

  Maisterintutkinto oli ollut saanut samanaikaisesti kun olin opiskelemassa
  Lappeenrannan Teknillisessa Yliopistossa.\\

& \\
\textbf{9\,/\,2015\,--\,nykyisyys:} &
\textmd{\textsl{Coursera.org kurssit (ei todistusta):}}

  \vspace{-10pt}
  \begin{itemize}
    \setlength\itemsep{-3pt}
    \item Game Design: Art and Concepts Specialization (nelj\"a kurssia)
    \item Introduction to Interactive Programming in Python (kaksi osaa)
    \item Introduction to Game Development (Unityn perehdytys)
    \item Python for Everybody (nelj\"a kurssia)
    \item Java Programming (kaksi kurssia)
  \end{itemize}
\\
& \\
\multicolumn{2}{l}{\cellcolor{myGray}\textbf{Ty\"okokemus:}}\\
& \\

\textbf{9\,/\,2012\,--\,9\,/\,2013:} &
\textmd{\textsl{Scientific Measuring Instruments Finland Oy, Lappeenranta,
  Projektip\"a\"allikk\"o}}

\textit{Yritys on vuonna 2011 perustettu ja on Ven\"aj\"an yrityksen (TKA)
    tyt\"aryhti\"o aikoa menn\"a Eurooppaan.}

  Olen ollut vastuullinen tehd\"a paperit\"oit\"a, j\"arjest\"a\"a kokouksia,
  hallita verkkosivuston kehittymisen ja muita teht\"avi\"a. Olen joskus ja
  onnistuneesti k\"aytt\"anyt Suomen kieli t\"oiss\"ani.
\\
&\\

  \textbf{5\,/\,2010\,--\,4\,/\,2011:} &
  \textmd{\textsl{Lappeenrannan teknillinen yliopisto, Nuorempi tutkija}}

  Suoritin tutkimus tietotekniikan osastossa, konen\"a\"on laboratoriossa.
  Tutkimuksen oli siit\"a k\"ayt\"ost\"a strukturoitujen valokuviot
  rekonstruoimaan kolmiulotteisen objektin muoto. T\"oiss\"ani k\"aytin
  Matlab-ohjelmisto ja C++ ohjelmointikieli Linuxin
  k\"aytt\"oj\"arjestelm\"ass\"a.
\\
&\\

  \textbf{6\,/\,2007\,--\,8\,/\,2007, 6\,/\,2006\,--\,8\,/\,2006:} &
  \textmd{\textsl{Internet-yritys ``Sampo.ru'', Petroskoi, Internetin luokan
    insin\"o\"ori}}

  \textit{Yritys on Petroskoin Internet-palveluntarjoaja.}

  Ty\"oskentelin asiakaspalvelussa, kassaty\"o ja toimistoty\"o. Autoin
  k\"aytt\"aji\"a Internetin luokassa, tulosta, kopiointi cd palava, skannaus,
  laminaatti ym. My\"os tein Internetin varauksen sopimuksia asiakkaille

\end{longtable}

\begin{longtable}{p{1.05\textwidth}}

\multicolumn{1}{l}{\cellcolor{myGray}\textbf{Taidot:}}\\\vspace{-5pt}

\emph{Kielet:} Erinomainen Englanti, melkein hyvin Suomi, perus Ranska ja
Ven\"aj\"a on \"aidinkieli.\\

\emph{IT:} Windows ja Linux k\"aytt\"oj\"arjestelm\"at (erinomainen), Mac OS X
  (basics), Office (Microsoft ja LibreOffice), Grafiikka (Inkscape and Gimp),
  CFD tools (Openfoam, Ansys Icem, Fluend, Paraview) (perusteet), Ohjelmointi
  (C/C++, Java, Python, \textsc{Matlab}, \LaTeX, HTML, Pascal).\\\ \\

\multicolumn{1}{l}{\cellcolor{myGray}\textbf{Harrastukset ja henkil\"okohtaiset
  ominaisuudet}}\\\vspace{-5pt}

  Viihdyn eri\"a urheilua, tanssiminen, pelej\"a (lautapelej\"a, tietokone).
  Minun mielenkiinnot sis\"alt\"av\"at ohjelmointi (erityisesti tietokone
  grafiikan alue), ruoanlaitto ja lukeminen (englanninkielen kirjoja auttavat
  levent\"a\"a sanavarastoni sek\"a olemassa viihdytt\"av\"a). Seurallinen,
  vastuullinen, tiimity\"ontekij\"a, tietokonetaitoinen, korkea analyyttisi\"a
  taitoja, nopea oppimaan, vahva ihmistenv\"aliset taidot, sopeutuva.
  Verenluovuttaja, l\"ahtien huhtikuu 2012.\\\ \\

\multicolumn{1}{l}{\cellcolor{myGray}\textbf{Suosittelijat:}}\\\vspace{-5pt}

\textbf{Tuomo Kauranne} / \textsl{Lappeenrannan teknillinen yliopisto,
  Matematiikan ja fysiikan laitos} / tutkijaopettaja, tuomo.kauranne [at]
  lut.fi, +358 40 530 0622\\
\textbf{Matylda Jablonska-Sabuka} / \textsl{Lappeenrannan teknillinen yliopisto,
  Matematiikan ja fysiikan laitos} / tutkijatohtori, matylda.jablonska-sabuka
  [at] lut.fi, +358 40 531 3041\\\ \\

\multicolumn{1}{l}{\cellcolor{myGray}\textbf{Julkaisut:}}\\\vspace{-15pt}

\nocite{Mar2:2008} \nocite{Mar:2008} \nocite{InvCamCal:2011}
\nocite{DetShaArt:2014} \nocite{ShaAreCal:2015}

\vspace{-30pt}

\renewcommand{\refname}{}
\bibliographystyle{unsrt}
\bibliography{ref}

\end{longtable}

\end{document}

