\documentclass[]{article}

\usepackage[none]{hyphenat}

%\usepackage{ifpdf}
%\ifpdf
%\usepackage[pdftex]{hyperref}
%\else
%\usepackage[hypertex]{hyperref}
%\fi

%\raggedright
%\linespread{1.2}
%\textheight = 700pt
%\textwidth = 370pt
%\hoffset = -37pt
%\voffset = -50pt
\pagestyle{empty}
%\righthyphenmin=55


\begin{document}

\begin{flushright}
	Ivan Martynov\\
  Huhtiniemenkatu 19 C 20\\
	53810 Lappeenranta, FINLAND\\
	+358 44 936 6589\\
	ivan.a.martynov [at] gmail.com\\
	\today
\end{flushright}

%\begin{flushleft}
%    FI-00520 Helsinki\\
%    FINLAND
%\end{flushleft}

%\begin{center}
%Gameplay Programmer Application
%\end{center}

\noindent
Dear Hiring Manager,
\bigskip

%\noindent
Please accept my cover letter for a programming Trainee position at Frozenbyte.
My primary goal is to become a gameplay or engine programmer. Both areas are
attractive and it is hard to tell which one is more interesting, however, I
currently feel more inclined towards the gameplay.

Therefore, I would be happy to have a chance to work as a trainee at Frozenbyte,
learn how your games are being built, help in the development of new games and
put my best effort to prove worth being part of the team aiming for a
permanent position in the future.

During my master studies at Lappeenranta University of Technology I have been
working with 3D point clouds, including the use of Delaunay triangulation
routine to compute certain features of those points (Betti numbers). My master
thesis can be found at http://www.doria.fi/handle/10024/42533.

Later on I have earned experience in image processing and camera/projector
calibration with the help of Matlab and C++ libraries. Recently I have created
a set of Matlab functions and routines for calculating shadow areas using
geographical elevation data.

Currently, most of my free time is devoted for learning computer graphics and
improving my C++ skills. Obviously, the learning process never ends and I am
eager to develop my skills specifically for games and continue with such an
experienced and independent game developer as Frozenbyte. I have started to put
various code samples at github: https://github.com/Tigrolik

Games have been a part of my life for a couple of decades and many of them
fascinated me to various levels. They have been gradually increasing my
curiosity about how games are made and thus ripening my decision to become a
game programmer. I have taken several university and Coursera.org courses
dedicated to games, went through various on-line material. Now, after studying
and practising the concepts of game design I have a better understanding how
the games are created.
%\medskip

With my wife we have enjoyed playing the Trine series and have found it to be a
lovely bonding experience when me being Pontius protecting her Zoya character
from the enemies while letting her shoot them or Amadeus giving her support for
reaching the higher places or solving puzzles. Besides, the graphics art
presented with stereoscopic effects and harmoniously accompanied by wonderful
music have aesthetically pleased us. We especially appreciated that the games
are available for the Linux platform so we did not need to switch to the Windows
OS in order to play a game.

My Finnish language skills are at the intermediate level (minulla on yleisten
kielitutkintojen suomen kielen keskitason tutkinton todistus), although, the
best way to check my actual skills of conversing in Finnish is via talking.

%\medskip

I am strongly motivated to be a game programmer and I am sure my skills will be
useful for programming tasks you games require to be done. I hope to have my
sleeves rolled up and start hands-on work with the Frozenbyte team.
Nowadays, I have a clear view in my life --- I want to make games.
\\\ \\

\noindent
I thank you in advance for your consideration.\\\ \\

\noindent Sincerely,\\
Ivan Martynov

\end{document}
