\documentclass[a4paper]{article}

\usepackage{longtable}

\usepackage[table]{xcolor}
\definecolor{myGray}{gray}{0.8}
\definecolor{emailGray}{gray}{0.1}

\usepackage{ifpdf}
\ifpdf
\usepackage[pdftex, colorlinks=true, urlcolor=emailGray]{hyperref}
\else
\usepackage[hypertex]{hyperref}
\fi


\hoffset=-50pt
\textwidth = 450pt
\voffset=-65pt
\textheight = 690pt

\linespread{1.1}

\begin{document}

\begin{center}
\Large{\bf{CURRICULUM VITAE}}
\end{center}

\begin{flushleft}
\textbf{Ivan Martynov}\\
Mob. 044 936 6589\\
E-mail: \href{ivan.a.martynov@gmail.com}{ivan.a.martynov@gmail.com}\\
Huhtiniemenkatu 19 C 20, 53810 Lappeenranta, Finland\\
Date of birth: 12.04.1982, id: 120482--165E\\
Citizenship: Finland, Russia
\end{flushleft}
\vspace{-10pt}

\begin{longtable}{p{0.12\textwidth} p{0.88\textwidth}}
  \textbf{Objective:} &
  I would like to have my analytical skills applied in the gaming world. The
  data analysis for defining the best heading directions for the modern games
  is a nice area for the professional growth. I am interested in sharing
  experience with the skillful team of Rovio experts and to test my competence
  as well as to enrich my skills.
\\
& \\
\multicolumn{2}{l}{\cellcolor{myGray}\textbf{Education:}}\\
& \\

\textbf{Jun~2012\,--\,present:} &
  \textmd{\textsl{Lappeenranta University of Technology, Faculty of Technology,
    Technomathematics and Technical Physics Department.}}

  \textit{Major and minor subjects:} Image processing, feature detection
  (e.\,g., clouds and shadows in geographic images). In my work I have used
  different approaches towards locating various properties of terrains:
  \vspace{-10pt}
  \begin{itemize}
    \setlength\itemsep{-3pt}
    \item[-]Analysis of spectral values of satellite images: applicable to
      the detection of clouds, water, vegetation and other terrain areas. Mostly
      used with Landsat satellite imagery;
    \item[-]Calculating shadows using the information about terrain elevation and
      the solar angles. Applicable to elevation geographical models, for
      example, Shuttle Radar Topography Mission (SRTM) data.
  \end{itemize}
  \vspace{-10pt}
  The methods allow us to evaluate which areas in any given terrain are to
  appear umbrous at a given time point.

  Doctoral degree in Technology (tentative time of completion: December 2017).
\\
&\\
\textbf{Sep~2006\,--\,Sep~2008:} &
  \textmd{\textsl{Lappeenranta University of Technology, Faculty of Technology,
    Technomathematics and Technical Physics Department.}}

  \textit{Major and minor subjects:} Academic writing in English, information
  technology, statistical analysis in modeling, evolutionary computation, etc.

  Supervisors: Prof. Heikki Haario and Adj. Prof. Tuomo Kauranne.
  Master's thesis: \emph{Computing the persistent homology of range images with
  alpha shapes.} The work is related to the processing of 3D point clouds with
  the analysis of the structure: depending on the points define the homology of
  the data. The devised methods can be used to compute Betti numbers and thus
  estimate the number of components in the point cloud, as well as the number
  of 1D and 2D holes.\\

&\\
\textbf{Sep~2002\,--\,Jun~2008:} &
  \textmd{\textsl{Petrozavodsk State University, Faculty of Mathematics,
    Topology and Geometry Department.}}

  \textit{Major and minor subjects:} Mathematical analysis, combinatorics,
  functional analysis, topology, differential equations, theoretical mechanics
  etc.

  Supervisor: Prof. Aleksandr Ivanov.
  Master's thesis: \emph{About free products homeomorphisms.} The work is
  developing a topological theory of homeomorphisms in specific topological
  structures.

  The master degree have been obtained in parallel with studying at Lappeenranta
  University of Technology.

\\

& \\
\textbf{Sep~2015\,--\,present:} &
\textmd{\textsl{Coursera.org courses (no certificate):}}

  \vspace{-10pt}
  \begin{itemize}
    \setlength\itemsep{-3pt}
    \item Game Design: Art and Concepts Specialization (four courses)
    \item Introduction to Interactive Programming in Python (two parts)
    \item Introduction to Game Development (introducing to Unity)
    \item Python for Everybody (four courses)
    \item Java Programming (two courses)
  \end{itemize}
\\
& \\
\multicolumn{2}{l}{\cellcolor{myGray}\textbf{Professional experience:}}\\
& \\

\textbf{Sep~2012\,--\,Sep~2013:} &
\textmd{\textsl{Scientific Measuring Instruments Finland Oy, Lappeenranta,
  Project manager}}

\textit{The company was established in 2011 as a daughter company of Russian
  company TKA with the intention to extend the activity of the latter to the
  European region.}

  I have been responsible for certain paper work, controlling the development of
  their web-site and other tasks. I have been occasionally and successfully
  using the Finnish language in my work.
\\
&\\

  \textbf{May~2010\,--\,Apr~2011:} &
  \textmd{\textsl{Lappeenranta University of Technology, Younger researcher}}

  I have been carrying out a research at the Information Technology department,
  Machine Vision laboratory. The project has been focused on the use of
  structured light patterns for a 3D reconstruction of the shape of an object.
  In my work I have been actively using the Matlab software and C++ programming
  language under Linux operating system.
\\
&\\

  \textbf{Jun~2007\,--\,Aug~2007, Jun~2006\,--\,Aug~2006:} &
  \textmd{\textsl{Internet company ``Sampo.ru'', Petrozavodsk, Engineer of an
    Internet class}}

  %Petrozavodsk,\\06 -- 08/2007 and 06 -- 08/2006 (tot. 6 months)}\\
  \textit{The company is an Internet provider in Petrozavodsk.}

  I have been working with customers, cash register and performing varying
  office work (e.\,g., carrying out print, copy, cd burning, scanning,
  laminating operations and similar functions). I have as well been helping
  users in an Internet class. For example, giving advice on the use of certain
  software or how to operate with a camera. Besides, I have been making
  agreements with customers for providing Internet from the company to their
  homes.
\end{longtable}

\begin{longtable}{p{1.05\textwidth}}

\multicolumn{1}{l}{\cellcolor{myGray}\textbf{Skills:}}\\\vspace{-5pt}

\emph{Languages:} Excellent English, intermediate Finnish, basic French,
native Russian.\\

\emph{Computer:} Windows and Linux OS (excellent), Mac OS X (basics), Office
tools (Microsoft and LibreOffice), Graphics (Inkscape and Gimp), CFD tools
(Openfoam, Ansys Icem, Fluend, Paraview) (basics), Programming (C/C++, Java,
Python, \textsc{Matlab}, \LaTeX, HTML, Pascal).\\\ \\

\multicolumn{1}{l}{\cellcolor{myGray}\textbf{Additional information or other
qualifications:}}\\\vspace{-5pt}

Communicative, responsible, team worker, computer literate, high analytical
skills, quick learner, strong interpersonal skills, adaptable. I enjoy various
sport, dancing, games (boardgames, computer). My interests comprise programming
(especially computer graphics area), cooking and reading (English language
books help to widen my vocabulary apart from being entertaining). Blood donor
since April 2012.\\\ \\

\multicolumn{1}{l}{\cellcolor{myGray}\textbf{References:}}\\\vspace{-5pt}

\textbf{Tuomo Kauranne} / \textsl{Lappeenranta University of Technology,
Mathematics and Physics department} /
adj. prof., lecturer, tuomo.kauranne [at] lut.fi, +358 40 530 0622\\
\textbf{Matylda Jablonska-Sabuka} / \textsl{Lappeenranta University of
Technology, Mathematics and Physics department} / post-doctoral researcher,
matylda.jablonska-sabuka [at] lut.fi, +358 40 531 3041\\\newpage

\multicolumn{1}{l}{\cellcolor{myGray}\textbf{Publications:}}\\\vspace{-15pt}

\nocite{Mar2:2008} \nocite{Mar:2008} \nocite{InvCamCal:2011}
\nocite{DetShaArt:2014} \nocite{ShaAreCal:2015}

\vspace{-30pt}

\renewcommand{\refname}{}
\bibliographystyle{unsrt}
\bibliography{ref}

\end{longtable}

\end{document}

